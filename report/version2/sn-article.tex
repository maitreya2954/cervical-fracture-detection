%%%%%%%%%%%%%%%%%%%%%%%%%%%%%%%%%%%%%%%%%%%%%%%%%%%%%%%%%%%%%%%%%%%%%
%%                                                                 %%
%% Please do not use \input{...} to include other tex files.       %%
%% Submit your LaTeX manuscript as one .tex document.              %%
%%                                                                 %%
%% All additional figures and files should be attached             %%
%% separately and not embedded in the \TeX\ document itself.       %%
%%                                                                 %%
%%%%%%%%%%%%%%%%%%%%%%%%%%%%%%%%%%%%%%%%%%%%%%%%%%%%%%%%%%%%%%%%%%%%%

%%\documentclass[referee,sn-basic]{sn-jnl}% referee option is meant for double line spacing

%%=======================================================%%
%% to print line numbers in the margin use lineno option %%
%%=======================================================%%

%%\documentclass[lineno,sn-basic]{sn-jnl}% Basic Springer Nature Reference Style/Chemistry Reference Style

%%======================================================%%
%% to compile with pdflatex/xelatex use pdflatex option %%
%%======================================================%%

%%\documentclass[pdflatex,sn-basic]{sn-jnl}% Basic Springer Nature Reference Style/Chemistry Reference Style

%%\documentclass[sn-basic]{sn-jnl}% Basic Springer Nature Reference Style/Chemistry Reference Style
\documentclass[pdflatex,sn-mathphys]{sn-jnl}% Math and Physical Sciences Reference Style
%%\documentclass[sn-aps]{sn-jnl}% American Physical Society (APS) Reference Style
%%\documentclass[sn-vancouver]{sn-jnl}% Vancouver Reference Style
%%\documentclass[sn-apa]{sn-jnl}% APA Reference Style
%%\documentclass[sn-chicago]{sn-jnl}% Chicago-based Humanities Reference Style
%%\documentclass[sn-standardnature]{sn-jnl}% Standard Nature Portfolio Reference Style
%%\documentclass[default]{sn-jnl}% Default
%%\documentclass[default,iicol]{sn-jnl}% Default with double column layout

%%%% Standard Packages
%%<additional latex packages if required can be included here>
%%%%

%%%%%=============================================================================%%%%
%%%%  Remarks: This template is provided to aid authors with the preparation
%%%%  of original research articles intended for submission to journals published 
%%%%  by Springer Nature. The guidance has been prepared in partnership with 
%%%%  production teams to conform to Springer Nature technical requirements. 
%%%%  Editorial and presentation requirements differ among journal portfolios and 
%%%%  research disciplines. You may find sections in this template are irrelevant 
%%%%  to your work and are empowered to omit any such section if allowed by the 
%%%%  journal you intend to submit to. The submission guidelines and policies 
%%%%  of the journal take precedence. A detailed User Manual is available in the 
%%%%  template package for technical guidance.
%%%%%=============================================================================%%%%

\jyear{2022}%

%% as per the requirement new theorem styles can be included as shown below
\theoremstyle{thmstyleone}%
\newtheorem{theorem}{Theorem}%  meant for continuous numbers
%%\newtheorem{theorem}{Theorem}[section]% meant for sectionwise numbers
%% optional argument [theorem] produces theorem numbering sequence instead of independent numbers for Proposition
\newtheorem{proposition}[theorem]{Proposition}% 
%%\newtheorem{proposition}{Proposition}% to get separate numbers for theorem and proposition etc.

\theoremstyle{thmstyletwo}%
\newtheorem{example}{Example}%
\newtheorem{remark}{Remark}%

\theoremstyle{thmstylethree}%
\newtheorem{definition}{Definition}%

\raggedbottom
%%\unnumbered% uncomment this for unnumbered level heads

\begin{document}

\title[Cervical Spine Fracture Detection and Localization]{Cervical Spine Fracture Detection and Localization}

%%=============================================================%%
%% Prefix	-> \pfx{Dr}
%% GivenName	-> \fnm{Joergen W.}
%% Particle	-> \spfx{van der} -> surname prefix
%% FamilyName	-> \sur{Ploeg}
%% Suffix	-> \sfx{IV}
%% NatureName	-> \tanm{Poet Laureate} -> Title after name
%% Degrees	-> \dgr{MSc, PhD}
%% \author*[1,2]{\pfx{Dr} \fnm{Joergen W.} \spfx{van der} \sur{Ploeg} \sfx{IV} \tanm{Poet Laureate} 
%%                 \dgr{MSc, PhD}}\email{iauthor@gmail.com}
%%=============================================================%%

\author{\fnm{Naman} \sur{Raghuvanshi}}\email{nvr5386@psu.edu}

\author{\fnm{Siddharth} \sur{Rayabharam}}\email{nqr5356@psu.edu}
\equalcont{These authors contributed equally to this work.}

\author{\fnm{Sumant} \sur{Suryawanshi}}\email{szs7220.psu.edu}
\equalcont{These authors contributed equally to this work.}


%%==================================%%
%% sample for unstructured abstract %%
%%==================================%%

\abstract{\textbf{Purpose:} To develop a deep learning model for the detection and localization of cervical spine fractures in the axial CT scans. 

\textbf{Methods:} We use the dataset consisting of cervical spine CT scans provided by the Radiological Society of North America (RSNA). The dataset consists of 3000 studies of individual patients for the training and testing combined. Out of these 3000 patient studies, 83 of these studies also contains segmentation data. Additionally, for x studies the dataset contains bounding box coordinates data. The dataset population is split into 90\% for training and 10\% for validation. We use EfficientNetV2\cite{DBLP:conf/icml/TanL21} to learn the segmentation, X model for detecting the fractures and Y model for drawing bounding boxes around the fracture area.

\textbf{Results:} Results need to be updated

\textbf{Conclusion:} The ability of a X model to detect and localize cervical spine fractures on axial CT radiographs with high sensitivity and specificity was demonstrated.
 }
% Fracture detection is a crucial part of a computer-aided medical system. In this project, we provide a unified technique for the detection and localization of cervical spine fractures from the axial CT scans of the cervical spine. This project is meant to speed up the diagnosis and is not a substitution of the medical experts. In this project, a model first segments the examination slices from C1 to C7 and detects the presence of the fracture in the slice. Moreover, another model has been proposed to localize the line-of-break using bounding boxes for easy visualization of the fracture.

%%================================%%
%% Sample for structured abstract %%
%%================================%%

% \abstract{\textbf{Purpose:} The abstract serves both as a general introduction to the topic and as a brief, non-technical summary of the main results and their implications. The abstract must not include subheadings (unless expressly permitted in the journal's Instructions to Authors), equations or citations. As a guide the abstract should not exceed 200 words. Most journals do not set a hard limit however authors are advised to check the author instructions for the journal they are submitting to.
% 
% \textbf{Methods:} The abstract serves both as a general introduction to the topic and as a brief, non-technical summary of the main results and their implications. The abstract must not include subheadings (unless expressly permitted in the journal's Instructions to Authors), equations or citations. As a guide the abstract should not exceed 200 words. Most journals do not set a hard limit however authors are advised to check the author instructions for the journal they are submitting to.
% 
% \textbf{Results:} The abstract serves both as a general introduction to the topic and as a brief, non-technical summary of the main results and their implications. The abstract must not include subheadings (unless expressly permitted in the journal's Instructions to Authors), equations or citations. As a guide the abstract should not exceed 200 words. Most journals do not set a hard limit however authors are advised to check the author instructions for the journal they are submitting to.
% 
% \textbf{Conclusion:} The abstract serves both as a general introduction to the topic and as a brief, non-technical summary of the main results and their implications. The abstract must not include subheadings (unless expressly permitted in the journal's Instructions to Authors), equations or citations. As a guide the abstract should not exceed 200 words. Most journals do not set a hard limit however authors are advised to check the author instructions for the journal they are submitting to.}

\keywords{Cervical Spine, Fracture Detection, Deep Learning}

%%\pacs[JEL Classification]{D8, H51}

%%\pacs[MSC Classification]{35A01, 65L10, 65L12, 65L20, 65L70}

\maketitle

\section{Introduction}\label{sec1}

Cervical spine injury is very common injury with more than 3 million cases per year that are being evaluated for cervical spine injury in North America\cite{Milby:2008tt}. In United States, more than 1 million patients with blunt force injury are suspected to suffer cervical spine injury\cite{Minja:2018ud}. Since cervical spine injury is associated with high morbidity and mortality, quick diagnosis of the injury is crucial. Any delay in diagnosis may result in devastating consequences for the patient. So, any additional aid to the radiologists can reduce the morbidity or mortality of the patient.

In recent years, a machine deep learning technique known as deep convolutional neural network (DCNN) has been applied to image recognition tasks. DCNN's are well suited for images. So, they have been used extensively in the field of medicine to classify medical images. 

In past few years, there have been many studies that have tried to use DCNN\cite{Olczak:2017aa}\cite{Kim:2018aa}\cite{Chung:2018aa} on medical radiographs. In these studies, the reference standard for the training and testing images was based on the assessment of human readers determining which were visible, only within a radiograph. Many radiologist fail to detect "occult fracture" because of the difficulty in detecting such fracture in a radiograph. These extraction methodologies could adversely influence the classification accuracy and occult fracture being assessed as a “non-fracture case”. A proficient algorithm may help identify and triage studies for the radiologist to review more urgently, helping to ensure faster diagnoses.

The purpose of our study is to develop an automated deep learning system for detecting cervical spine fractures using CT a gold standard annotated by radiologists, and to evaluate the diagnostic performance inclusive of the experienced readers in detecting cervical spine fractures on radiographs. 

\section{Methodology}\label{sec2}

\subsection{Dataset and Study Population}\label{subsec2}

We use the dataset consisting of cervical spine CT scans provided by the Radiological Society of North America (RSNA). The dataset we are using is made up of roughly 3000 CT studies, from twelve locations and across six continents. Spine radiology specialists have provided annotations to indicate the presence, vertebral level and location of any cervical spine fractures. Each radiology study consists of multiple dcm files. A dcm file follows the Digital Imaging and Communications in Medicine (DICOM) format. It is the standard format used for storing medical images and related metadata. It dates back to 1983, although it has been revised many times. Out of these 3000 patient studies, 83 of these studies also contains segmentation data. Additionally, for x studies the dataset contains bounding box coordinates data. The dataset population is split into 90\% for training and 10\% for validation. The distribution of fractured and non-fractured dataset in the dataset is shown in the below two graphs.

Bar graph and pie graph for the dataset.

\subsubsection{Segmentation}\label{subsubsec1}

Segmentation involves classification of CT radiographs into C1 to C7 labels. Using the 83 studies which have segmentation data. We train Random Forest classifier\cite{SEG:RandForest} and EfficientNetV2\cite{DBLP:conf/icml/TanL21}. Correct vertebrae labels are essential when training the fracture prediction model. First we extract vertebrae targets from 83 segmentation files.

\begin{table}[!h]
\centering
\begin{tabular}{|c|c|c|}
\hline
\textbf{Vetebrae} & \textbf{\begin{tabular}[c]{@{}c@{}}Random Forest \\ Classifier\end{tabular}} & \textbf{EfficientNetV2} \\ \hline
C1 & 0.8891 & 0.9621 \\ \hline
C2 & 0.8898 & 0.9409 \\ \hline
C3 & 0.8962 & 0.9559 \\ \hline
C4 & 0.8952 & 0.9605 \\ \hline
C5 & 0.8893 & 0.9482 \\ \hline
C6 & 0.8713 & 0.9450 \\ \hline
C7 & 0.8576 & 0.9465 \\ \hline
Overall & 0.8841 & 0.9513 \\ \hline
\end{tabular}
\caption[]
	{\centering\small Accuracies of the models for the segmentation}
\end{table}

From Table 1, we can see that EfficientNet performs well compared to the Random Forest Classifier. EfficientNetV2 is a new family of convolutional networks that have faster training speed and better parameter efficiency than previous models. To develop these models, a combination of training-aware neural architecture search and scaling was used to jointly optimize training speed and parameter efficiency. The models were searched from the search space enriched with new ops such as Fused-MBConv\cite{Fused-MBConv}. Experiments show that EfficientNetV2 models train much faster than state-of-the-art models while being up to 6.8x smaller. With progressive learning, EfficientNetV2 significantly outperforms previous models on ImageNet and CIFAR/Cars/Flowers datasets. By pretraining on the same ImageNet21k, EfficientNetV2 achieves 87.3\% top-1 accuracy on ImageNet ILSVRC2012, outperforming the recent ViT by 2.0\% accuracy while training 5x-11x faster using the same computing resources

\subsubsection{Fracture Detection}\label{subsubsec2}

\subsubsection{Localization}\label{subsubsec3}

\section{Results}\label{sec3}

\section{Discussion}\label{sec12}

Discussions should be brief and focused. In some disciplines use of Discussion or `Conclusion' is interchangeable. It is not mandatory to use both. Some journals prefer a section `Results and Discussion' followed by a section `Conclusion'. Please refer to Journal-level guidance for any specific requirements. 

\section{Conclusion}\label{sec13}

Conclusions may be used to restate your hypothesis or research question, restate your major findings, explain the relevance and the added value of your work, highlight any limitations of your study, describe future directions for research and recommendations. 

In some disciplines use of Discussion or 'Conclusion' is interchangeable. It is not mandatory to use both. Please refer to Journal-level guidance for any specific requirements. 

\backmatter

\begin{appendices}

\section{Section title of first appendix}\label{secA1}

An appendix contains supplementary information that is not an essential part of the text itself but which may be helpful in providing a more comprehensive understanding of the research problem or it is information that is too cumbersome to be included in the body of the paper.

%%=============================================%%
%% For submissions to Nature Portfolio Journals %%
%% please use the heading ``Extended Data''.   %%
%%=============================================%%

%%=============================================================%%
%% Sample for another appendix section			       %%
%%=============================================================%%

%% \section{Example of another appendix section}\label{secA2}%
%% Appendices may be used for helpful, supporting or essential material that would otherwise 
%% clutter, break up or be distracting to the text. Appendices can consist of sections, figures, 
%% tables and equations etc.

\end{appendices}

%%===========================================================================================%%
%% If you are submitting to one of the Nature Portfolio journals, using the eJP submission   %%
%% system, please include the references within the manuscript file itself. You may do this  %%
%% by copying the reference list from your .bbl file, paste it into the main manuscript .tex %%
%% file, and delete the associated \verb+\bibliography+ commands.                            %%
%%===========================================================================================%%

\bibliography{sn-bibliography}% common bib file
%% if required, the content of .bbl file can be included here once bbl is generated
%\input sn-report.bib

%% Default %%
%%\input sn-sample-bib.tex%

\end{document}
